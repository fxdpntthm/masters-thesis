\section{Background Work}\label{sec:background}
% (10 mins)
\begin{frame}[c]
  \frametitle{Background Work}
  Hinley-Milner (\HM{}) type system
  Algorithm $\M$\citep{damas_principal_1982}
  Algorithm $\W$\citep{lee_proofs_1998}
\end{frame}

\begin{frame}[c]
  \frametitle{Background Work}
  Curry Howard Correspondence: \\
  \HM type system is equivalent to second order propositional logic.\\
  Propostions are not Resources.
\end{frame}


\begin{frame}[c]
  \frametitle{Background Work}
  Substructural Logics:
  Make structural rules explicit
\end{frame}

\begin{frame}[c]
  \frametitle{Background Work}
  Linear Logic\citep{girard_linear_1987, wadler_taste_1993}\\
  restrict weakening and contraction\\
  Propostions act like resources\\
  \begin{itemize}
  \item Modality $\oc$
  \item Additive pair: $\with$
  \item Multiplicative pair: $\otimes$ $\rightspoon$
  \item $\oplus$
  \end{itemize}
\end{frame}

\begin{frame}[c]
  \frametitle{Background Work}
    Theory of qualified types: Polymorphism using Predicates \HM{}\citep{jones_theory_1994}
\end{frame}


\begin{frame}[c]
  \frametitle{Background Work}
  Quill: Linear Logic using with Qualified Types\citep{morris_best_2016}
\end{frame}

\begin{frame}[c]
  \frametitle{Background Work}
  Logic of Bunched Implications (\BI{})\citep{ohearn_logic_1999}
  \begin{itemize}
  \item Sharing implication $\shimp$
  \item Separating implication $\sepimp$
  \end{itemize}
\end{frame}


%%% Local Variables:
%%% mode: latex
%%% TeX-master: t
%%% End:
