\chapter{Proofs}
\begin{theorem}[Soundness of $\vdashs$]\label{thm:soundness-syntax-directed}
   If $P \mid \Gamma \vdashs M:\tau$ then $P \mid \Gamma \vdash M:\tau$
\end{theorem}
\begin{proof}\label{prf:soundness-syntax-directed}
  Proof by structural induction on the derivation of $P \mid \Gamma \vdashs M:\tau$.

  \begin{itemize}
  \item{Case [VAR$^s$].}
    By induction hypothesis we have a derivation of $P \mid x:\sigma \vdash x:\sigma$ by [VAR] rule.
    We proceed by repeated application of [$\forall$E] as $(Q => \tau) \sqsubseteq \sigma$, and
    we construct a derivation of $P \mid x:\sigma \vdash x: Q => \tau$. Then as $P => Q$ we can
    repeatedly apply [$=>$E] to construct a derivation of $P \mid x:\sigma \vdash x:\tau$.
    Finally, depending on whether the bindings are in sharing with or separate from
    $x$ we repeatedly apply [WKN-SH] or [WKN-UN] respectively for all the bindings in $\Gamma$ to construct
    a derivation of $P \mid \Gamma \sqcup x:\sigma \vdash x:\tau$.
  \item{Case [$\rightarrow$I$^s$].}
    By induction hypothesis we have a derivation of $P \mid \Gamma \varoplus x:\tau \vdash M:\tau'$.
    We apply [$\rightarrow$I] and reuse the derivations for $\ShFun{\phi}$ and $\Gamma \geq \phi$  to
    construct a derivation of $P \mid \Gamma \vdash \lambda^{\alpha}x. M: \phi \tau \tau'$.
  \item{Case [$\sepimp$I$^s$].}
    By induction hypothesis we have a derivation of $P \mid \Gamma \circledast x:\tau \vdash M:\tau'$.
    Similar to previous case, we apply [$\sepimp$I] and reuse the derivations for $\SeFun{\phi}$ and $\Gamma \geq \phi$  to
    construct a derivation of $P \mid \Gamma \vdash \lambda^{*}x. M: \phi \tau \tau'$.
  \item{Case [App$^s$].}
    By induction hypothesis we have derivations of $P \mid \Gamma \circledast \Delta \vdash M: \phi v \tau$ and
    $P \mid \Gamma' \circledast \Delta \vdash: N:v$ we check for $\texttt{Used}(\Gamma) = \texttt{Used}(\Gamma')$ and if it is true
    and apply [$\rightarrow$E] or check for $\texttt{Used}(\Gamma)\#\texttt{Shared}(\Gamma') \wedge \texttt{Shared}(\Gamma')\#\texttt{Used}(\Gamma)$
    and if it is true we apply [$\sepimp$E] to reuse derivations of $\ShFun{\phi}$ or $\SeFun{\phi}$ respectively to construct the derivation
    of $P \mid (\Gamma \varoplus \Gamma') \circledast \Delta \vdash M N:\tau$ or $P \mid (\Gamma \circledast \Gamma') \circledast \Delta \vdash M N:\tau$.
  \item{Case [Let$^s$].}
    By induction hypothesis have a derivation of $P \mid \Gamma \circledast \Delta \vdash M: \tau$ and $P \mid \Gamma' \sqcup x:\tau \circledast \Delta \vdash N:\tau$
    Applying [$\forall$I] and [$=>$I] on the first hypothesis we derive $\emptyset \mid \Gamma \circledast \Delta \vdash M: \sigma$. Now by applying
    the [LET] rule and reusing $P \vdash \Delta\ \texttt{un}$ we construct the derivation of
    $P \mid \Gamma \sqcup \Gamma' \circledast \Delta \vdash \Let{x}{M}{N}:\tau$.
  \end{itemize}
\end{proof}

\begin{theorem}[Completeness of $\vdashs$]\label{thm:completeness-syntax-directed}
  If $P \mid \Gamma \vdash M:\tau$ then
  $\exists ! Q,\tau. Q \mid \Gamma \vdashs M:\tau$
  and $(P \mid \sigma) \sqsubseteq \texttt{Gen}(\Gamma, Q => \tau)$
\end{theorem}
\begin{proof}
  \TODO{some induction here}.
\end{proof}

\begin{theorem}[Soundness of $\mathcal{M}$.]
   If $\mathcal{M}(S, X; \Gamma \vdash M : \tau) = P, S', \Sigma$ then $S' P | S' (\Gamma\mid_{\Sigma}) \vdash M : S' \tau$
\end{theorem}
\begin{proof}
  Proof by induction on structure of $M$
  \begin{itemize}
  \item Case 1. $x$
  \item Case 2. $\lambda^{*} x. M$
  \item Case 3. $\lambda ^{\alpha}x. M$
  \item Case 4. $M\ N$
  \item Case 5. $\texttt{let}\ x = M\ \texttt{in}\ N$
  \end{itemize}
\end{proof}

\begin{theorem}[Completeness of $\M$.]
  If $S$ is a substitution ans X is a set of type variables such that
  $P \mid S\Gamma \vdashs M: S\tau$, and $S|_X = id$ then $\M(id, X; \Gamma \vdash M:\tau) = Q, S', \Sigma$, such that
  $P => S\tau \sqsubset \texttt{GenI}(S'\Gamma, S' Q => S' \tau)$
\end{theorem}
\begin{proof}
\TODO{some more induction here}
\end{proof}

\begin{theorem}[Principal types.]
  If $P_0 \mid  \Gamma \vdash M : \sigma_0$ and $P_1 \mid  \Gamma \vdash M : \sigma_1$ then there is some $\sigma$ such that
  $\emptyset \mid \Gamma \vdash M : \sigma$ and $(P_0 | \sigma_0) \subseteq \sigma$, and $(P_1 | \sigma_1) \subseteq \sigma$.
\end{theorem}
\begin{proof}
\TODO{some more induction here. Makes use of theorems \ref{thm:soundness-syntax-directed} and \ref{thm:completeness-syntax-directed}}
\end{proof}



\TODO{
  \begin{theorem}[Progress.]
    If $M:\tau$ then either  $M \leadsto M'$ or $M$ is a value.
  \end{theorem}
\begin{proof}
  we have to define $\beta \eta$ rules.
\end{proof}

\begin{theorem}[Preservation.]
  If $M:\tau$ and $M \leadsto M'$ then $M':\tau$
\end{theorem}
\begin{proof}
\end{proof}
}


%%% Local Variables:
%%% mode: latex
%%% TeX-master: "../thesis-ku"
%%% End:
